\section{بررسی کارهای مرتبط پیشین}
\label{sec:survey}
تمرکز اصلی این پژوهش بر ارائه‌ی روش زمان‌بندی آگاه از دمای سطح تراشه و قابلیت اطمینان در سامانه‌های بحرانی-مختلط چند هسته‌ای همگن است. با توجه به آنکه سامانه‌های نهفته‌ی بحرانی-مختلط نوع ارتقاء یافته‌ای از سامانه‌های نهفته‌ی بحرانی-ایمن هستند\cite{Baruah2012b}، در این بخش سعی می‌شود به بررسی برخی کارهای مرتبط با این پژوهش و سامانه‌های بحرانی-مختلط در حوزه‌ی سامانه‌های بحرانی-ایمن نیز پرداخته شود. به همین منظور در ابتدا به بررسی کارهای پیشین در حوزه‌ی نگاشت و زمان‌بندی وظایف در سامانه‌های بحرانی-مختلط پرداخته می‌شود سپس به بررسی کار‌های پیشین در حوزه‌ی تحمل‌پذیری اشکال در سامانه‌های بحرانی-ایمن و بحرانی-مختلط پرداخته می‌شود و در انتها به بررسی پژوهش‌های حوزه‌ی مدیریت دما و سقف توان پرداخته می‌شود.

\subsection{نگاشت و زمان‌بندی وظایف در سامانه‌های بحرانی-مختلط}

\subsection{تحمل‌پذیری اشکال در سامانه‌های بحرانی-مختلط}

\subsection{مدیریت سقف توان مصرفی و حداکثر دمای سطح تراشه}
\begin{table}[b!]
\caption{جمع‌بندی و مقایسه روش‌های پیشین }
	\centering
\begin{tabular}{||c|c|c|c|c|c||}
	\hline
	روش                                                                              & بستر        & مدیریت دما/سقف توان & مدل سامانه& تحمل‌پذیری اشکال    & مدل وظایف         \\ \hline
	\cite{Baruah2008}\cite{Vestal2007}\cite{Baruah2015}\cite{Ekberg2014}             & تک‌هسته‌ای  & \xmark              & بحرانی-مختلط                                                                       & \xmark                  & پراکنده      \\ \hline
	\cite{Su2013}                                                                    & تک‌هسته‌ای  & \xmark              & بحرانی-مختلط                                                                      & \xmark                  & متناوب            \\ \hline
	\cite{Li2012}                                                                    & چند‌هسته‌ای & \xmark              &بحرانی-مختلط                                                                      & \xmark                  & پراکنده     \\ \hline
		روش پیشنهادی                                                                     & چندهسته‌ای  & \cmark              & بحرانی-مختلط                                                                        & افزونگی چندپیمانه‌ای & گراف-متناوب              \\ \hline
\end{tabular}
\label{tab:survey}
\end{table}




 
